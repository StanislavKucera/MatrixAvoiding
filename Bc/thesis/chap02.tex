\chapter{An algorithm for testing pattern-avoidance of a general pattern}
We begin with a very basic algorithm that we improve a lot to get a fast algorithm for testing avoidance of a general pattern.

\section{Sketch of a brute force algorithm}
Let $L=\{l_1,l_2,\cdots,l_{w+h}\}$ be the lines (rows and columns) of the pattern $P$. Partial mapping of lines of $P$ is a function $f$ from $L'\subseteq L$ to lines of the big matrix $M$ satisfying two conditions: 
%The basic algorithm we use goes as follows. It takes a line $l$ (row or column) of the pattern and for each line $L$ of the big matrix it decides whether the pattern can be mapped there. It can, if three conditions are all met at once:
\begin{itemize}
%\item Both $l$ and $L$ are rows or they are both columns.
%\item If $l'$ is a line of $P$ parallel to $l$ that has been mapped previously to line $L'$, we want $l<l'\Leftrightarrow L<L'$. This means we want the mapped lines to be in the same order as they were in the pattern.
%\item If $l'$ is a line of the pattern orthogonal to $l$ that has been mapped previously to line $L'$ and there is a one-entry at the intersection of $l$ and $l'$, there has to be one-entry at the intersection of $L$ and $L'$.
\item Both $l'\in L'$ and $f(l')$ are rows or they are both columns.
\item If $l'\in L'$ and $l''\in L'$ are both rows or columns and $l'<l''$, then $f(l')<f(l'')$. This means partial mapping keeps the order of the lines.
\item If $l'\in L'$ is a row of $P$ and $l''\in L'$ is a column of $P$ and there is a one-entry at the intersection of $l'$ and $l''$, then there is a one-entry at the intersection of $f(l')$ and $f(l'')$.
\end{itemize}
The basic algorithm we use goes as follows. First it maps $l_1$ to all possible lines of $M$, creating partial mappings of $\{l_1\}\subseteq L$. For $k=2,\cdots,w+h$ it takes each partial mapping from previous iteration and extends it by adding line $l_k$ to the partial mapping in all possible ways. If we manage to map all the lines of $P$, then $M$ does not avoid it and if at some point there are no partial mappings to extend it means $M$ avoids $P$.

Note that we do not use even the slightest heuristics like not to map a line having $k$ one-entries to a line with fewer one-entries. The fact we made a useless mapping will be discovered at the time we try to map the $k$-th orthogonal line (or maybe earlier). This leads to possibly very large number of partial mappings and our next goal will be to reduce this number significantly.

\section{Equivalent mappings}
There are a lot of possible heuristics we can check before adding a line to a partial mapping, but the most fundamental operation we do to decrease the number of found mapping is saying that two mapping are equivalent.

The idea behind it is very basic. There is no need to remember two different mappings if they can be both extended exactly the same way as our function is only supposed to check whether a pattern can be mapped to a big matrix not to find all such mappings.

A \textbf{level} of a partial mapping is the number of lines mapped by the mapping.

We call a line $l$ of a pattern \textbf{important} in a partial mapping if one of the conditions is met:
\begin{itemize}
\item An adjacent line of the pattern has not been mapped yet.
\item There is a one-entry on the line $l$ at the intersection with line $l'$ that has not been mapped yet.
\end{itemize}.
Otherwise the line is \textbf{unimportant} in the mapping.

At the beginning, when no line is mapped, all lines are important. After some lines get mapped a line can become unimportant in the partial mapping as all lines that bound in are in the mapping as well. If a line is unimportant in a partial mapping of some level, it will stay unimportant in all extensions of the mapping we can find.

We say two partial mappings of the same level are \textbf{equivalent} if all important lines in the mapping of that level are mapped to the same lines of the big matrix in both mappings.

Picture and use...